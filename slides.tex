\documentclass{beamer}

\usepackage{color}
\newcommand{\red}[1]{\textcolor{red}{#1}}
\newcommand{\blue}[1]{\textcolor{blue}{#1}}
\usetheme{Berlin}
\setbeamercolor{block title}{fg=white,bg=blue!75!black}
\setbeamercolor{block body}{fg=black,bg=blue!10}


\title{Géometrie Différentielle}
\author{Author}

\begin{document}
\frame{\maketitle}
\section{Introduction}
\begin{frame}
  \frametitle{Droit ou courbé ?}
  \begin{figure}
    \centering
    \includegraphics[width=80mm]{asset1.eps}
  \end{figure}
\end{frame}

\begin{frame}
  \frametitle{Plate ou courbée ?}
  \begin{figure}
    \centering
    \includegraphics[width=80mm]{horizon.eps}
  \end{figure}
\end{frame}

\begin{frame}
  \frametitle{Plat ou courbé ?}
  \begin{figure}
    \centering
    \includegraphics[width=80mm]{universe.eps}
  \end{figure}
\end{frame}

\begin{frame}
  \frametitle{Plan}
  \begin{enumerate}
    \item Introduction
    \item Problématique : Comment savoir être courbé ?
    \item Découverte de Gauss 
    \item Developpements et Applications
  \end{enumerate}
\end{frame}


\section{Problématique} 
\begin{frame}
  \frametitle{La Terre est-elle vraiment courbée ?}
  \begin{figure}
    \centering
    \includegraphics[width=70mm]{earth.eps}
  \end{figure}
\end{frame}
\begin{frame}
  \frametitle{La Terre est-elle vraiment courbée ?}
  \begin{figure}
    \centering
    \includegraphics[width=80mm]{curve.eps}
  \end{figure}
\end{frame}
\begin{frame}
  \frametitle{La Terre est-elle vraiment courbée ?}
  \begin{figure}
    \centering
    \includegraphics[width=80mm]{map-line.eps}
  \end{figure}
\end{frame}
\begin{frame}
  \frametitle{La Terre est-elle vraiment courbée ?}
  \begin{figure}
    \centering
    \includegraphics[width=80mm]{map-distance.eps}
  \end{figure}
\end{frame}

\section{Theorema Egregium}
\begin{frame}
  \frametitle{Theorema Egregium}
  Theorema Egregium (Théorème remarquable)
  \vskip5mm
  \begin{minipage}{0.4\linewidth}
    \includegraphics[width=30mm]{gauss.eps} \\
    Carl Friedrich Gauss
  \end{minipage}
  \begin{minipage}{0.55\linewidth}
    \vspace{-1cm}
    La \red{courbure intrinsèque} d'une surface est déterminée uniquement par des \blue{mesures internes} à la surface.
  \end{minipage}
\end{frame}

\section{Developpements et Applications}
\begin{frame}
  \frametitle{Théorie de la relativité}
    Une phisique dans espace-temps en dimension 4 (3D + temps)
  \begin{itemize}
    \item La masse courbe l'espace-temps 
    \item Tous les objets bouge tout droit dans l'espace-temps courbé
  \end{itemize}
  \centering
  \includegraphics[width=5cm]{gravitational-lens.eps}
  \begin{itemize}
    \item La forme de l'univers ?
    \item Trou noir
    \item GPS
  \end{itemize}
\end{frame}

\begin{frame}
  \frametitle{Science des données}
  \begin{itemize}
    \item Hypothèse de la variété
    \item Géometrie de l'information
  \end{itemize}
\end{frame}

\begin{frame}
  Quels liens voyez-vous entre cette théorie et votre domaine ? 
\end{frame}
\end{document}
